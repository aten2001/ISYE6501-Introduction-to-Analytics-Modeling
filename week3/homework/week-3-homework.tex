% Options for packages loaded elsewhere
\PassOptionsToPackage{unicode}{hyperref}
\PassOptionsToPackage{hyphens}{url}
%
\documentclass[
]{article}
\usepackage{lmodern}
\usepackage{amssymb,amsmath}
\usepackage{ifxetex,ifluatex}
\ifnum 0\ifxetex 1\fi\ifluatex 1\fi=0 % if pdftex
  \usepackage[T1]{fontenc}
  \usepackage[utf8]{inputenc}
  \usepackage{textcomp} % provide euro and other symbols
\else % if luatex or xetex
  \usepackage{unicode-math}
  \defaultfontfeatures{Scale=MatchLowercase}
  \defaultfontfeatures[\rmfamily]{Ligatures=TeX,Scale=1}
\fi
% Use upquote if available, for straight quotes in verbatim environments
\IfFileExists{upquote.sty}{\usepackage{upquote}}{}
\IfFileExists{microtype.sty}{% use microtype if available
  \usepackage[]{microtype}
  \UseMicrotypeSet[protrusion]{basicmath} % disable protrusion for tt fonts
}{}
\makeatletter
\@ifundefined{KOMAClassName}{% if non-KOMA class
  \IfFileExists{parskip.sty}{%
    \usepackage{parskip}
  }{% else
    \setlength{\parindent}{0pt}
    \setlength{\parskip}{6pt plus 2pt minus 1pt}}
}{% if KOMA class
  \KOMAoptions{parskip=half}}
\makeatother
\usepackage{xcolor}
\IfFileExists{xurl.sty}{\usepackage{xurl}}{} % add URL line breaks if available
\IfFileExists{bookmark.sty}{\usepackage{bookmark}}{\usepackage{hyperref}}
\hypersetup{
  pdftitle={Homework3},
  pdfauthor={Chen Yi-Ju(Ernie)},
  hidelinks,
  pdfcreator={LaTeX via pandoc}}
\urlstyle{same} % disable monospaced font for URLs
\usepackage[margin=1in]{geometry}
\usepackage{color}
\usepackage{fancyvrb}
\newcommand{\VerbBar}{|}
\newcommand{\VERB}{\Verb[commandchars=\\\{\}]}
\DefineVerbatimEnvironment{Highlighting}{Verbatim}{commandchars=\\\{\}}
% Add ',fontsize=\small' for more characters per line
\usepackage{framed}
\definecolor{shadecolor}{RGB}{248,248,248}
\newenvironment{Shaded}{\begin{snugshade}}{\end{snugshade}}
\newcommand{\AlertTok}[1]{\textcolor[rgb]{0.94,0.16,0.16}{#1}}
\newcommand{\AnnotationTok}[1]{\textcolor[rgb]{0.56,0.35,0.01}{\textbf{\textit{#1}}}}
\newcommand{\AttributeTok}[1]{\textcolor[rgb]{0.77,0.63,0.00}{#1}}
\newcommand{\BaseNTok}[1]{\textcolor[rgb]{0.00,0.00,0.81}{#1}}
\newcommand{\BuiltInTok}[1]{#1}
\newcommand{\CharTok}[1]{\textcolor[rgb]{0.31,0.60,0.02}{#1}}
\newcommand{\CommentTok}[1]{\textcolor[rgb]{0.56,0.35,0.01}{\textit{#1}}}
\newcommand{\CommentVarTok}[1]{\textcolor[rgb]{0.56,0.35,0.01}{\textbf{\textit{#1}}}}
\newcommand{\ConstantTok}[1]{\textcolor[rgb]{0.00,0.00,0.00}{#1}}
\newcommand{\ControlFlowTok}[1]{\textcolor[rgb]{0.13,0.29,0.53}{\textbf{#1}}}
\newcommand{\DataTypeTok}[1]{\textcolor[rgb]{0.13,0.29,0.53}{#1}}
\newcommand{\DecValTok}[1]{\textcolor[rgb]{0.00,0.00,0.81}{#1}}
\newcommand{\DocumentationTok}[1]{\textcolor[rgb]{0.56,0.35,0.01}{\textbf{\textit{#1}}}}
\newcommand{\ErrorTok}[1]{\textcolor[rgb]{0.64,0.00,0.00}{\textbf{#1}}}
\newcommand{\ExtensionTok}[1]{#1}
\newcommand{\FloatTok}[1]{\textcolor[rgb]{0.00,0.00,0.81}{#1}}
\newcommand{\FunctionTok}[1]{\textcolor[rgb]{0.00,0.00,0.00}{#1}}
\newcommand{\ImportTok}[1]{#1}
\newcommand{\InformationTok}[1]{\textcolor[rgb]{0.56,0.35,0.01}{\textbf{\textit{#1}}}}
\newcommand{\KeywordTok}[1]{\textcolor[rgb]{0.13,0.29,0.53}{\textbf{#1}}}
\newcommand{\NormalTok}[1]{#1}
\newcommand{\OperatorTok}[1]{\textcolor[rgb]{0.81,0.36,0.00}{\textbf{#1}}}
\newcommand{\OtherTok}[1]{\textcolor[rgb]{0.56,0.35,0.01}{#1}}
\newcommand{\PreprocessorTok}[1]{\textcolor[rgb]{0.56,0.35,0.01}{\textit{#1}}}
\newcommand{\RegionMarkerTok}[1]{#1}
\newcommand{\SpecialCharTok}[1]{\textcolor[rgb]{0.00,0.00,0.00}{#1}}
\newcommand{\SpecialStringTok}[1]{\textcolor[rgb]{0.31,0.60,0.02}{#1}}
\newcommand{\StringTok}[1]{\textcolor[rgb]{0.31,0.60,0.02}{#1}}
\newcommand{\VariableTok}[1]{\textcolor[rgb]{0.00,0.00,0.00}{#1}}
\newcommand{\VerbatimStringTok}[1]{\textcolor[rgb]{0.31,0.60,0.02}{#1}}
\newcommand{\WarningTok}[1]{\textcolor[rgb]{0.56,0.35,0.01}{\textbf{\textit{#1}}}}
\usepackage{graphicx,grffile}
\makeatletter
\def\maxwidth{\ifdim\Gin@nat@width>\linewidth\linewidth\else\Gin@nat@width\fi}
\def\maxheight{\ifdim\Gin@nat@height>\textheight\textheight\else\Gin@nat@height\fi}
\makeatother
% Scale images if necessary, so that they will not overflow the page
% margins by default, and it is still possible to overwrite the defaults
% using explicit options in \includegraphics[width, height, ...]{}
\setkeys{Gin}{width=\maxwidth,height=\maxheight,keepaspectratio}
% Set default figure placement to htbp
\makeatletter
\def\fps@figure{htbp}
\makeatother
\setlength{\emergencystretch}{3em} % prevent overfull lines
\providecommand{\tightlist}{%
  \setlength{\itemsep}{0pt}\setlength{\parskip}{0pt}}
\setcounter{secnumdepth}{-\maxdimen} % remove section numbering

\title{Homework3}
\author{Chen Yi-Ju(Ernie)}
\date{2020/6/4}

\begin{document}
\maketitle

\hypertarget{question-7.1}{%
\subsection{Question 7.1}\label{question-7.1}}

\hypertarget{describe-a-situation-or-problem-from-your-job-everyday-life-current-events-etc.-for-which-exponential-smoothing-would-be-appropriate.-what-data-would-you-need-would-you-expect-the-value-of-ux3b1-the-first-smoothing-parameter-to-be-closer-to-0-or-1-and-why}{%
\subsubsection{Describe a situation or problem from your job, everyday
life, current events, etc., for which exponential smoothing would be
appropriate. What data would you need? Would you expect the value of α
(the first smoothing parameter) to be closer to 0 or 1, and
why?}\label{describe-a-situation-or-problem-from-your-job-everyday-life-current-events-etc.-for-which-exponential-smoothing-would-be-appropriate.-what-data-would-you-need-would-you-expect-the-value-of-ux3b1-the-first-smoothing-parameter-to-be-closer-to-0-or-1-and-why}}

I would consider oil price being a good situation to use exponential
smoothing. α would be somewhere closer to 1 than 0 because the oil
price(under current situations) are known to have big fluncuations due
to random events happening.

\hypertarget{question-7.2}{%
\subsection{Question 7.2}\label{question-7.2}}

\hypertarget{using-the-20-years-of-daily-high-temperature-data-for-atlanta-july-through-october-from-question-6.2-file-temps.txt-build-and-use-an-exponential-smoothing-model-to-help-make-a-judgment-of-whether-the-unofficial-end-of-summer-has-gotten-later-over-the-20-years.}{%
\subsubsection{Using the 20 years of daily high temperature data for
Atlanta (July through October) from Question 6.2 (file temps.txt), build
and use an exponential smoothing model to help make a judgment of
whether the unofficial end of summer has gotten later over the 20
years.}\label{using-the-20-years-of-daily-high-temperature-data-for-atlanta-july-through-october-from-question-6.2-file-temps.txt-build-and-use-an-exponential-smoothing-model-to-help-make-a-judgment-of-whether-the-unofficial-end-of-summer-has-gotten-later-over-the-20-years.}}

\hypertarget{my-answer-is-no.-according-to-exponential-smoothing-summer-has-not-gotten-latter-over-the-20-years.this-is-the-proccess-for-proving-it.}{%
\subsubsection{\texorpdfstring{My answer is No.~According to exponential
smoothing, summer has not gotten latter over the 20 years.This is the
proccess for proving
it.}{My answer is No.~According to exponential smoothing, summer has not gotten latter over the 20 years.This is the proccess for proving it. }}\label{my-answer-is-no.-according-to-exponential-smoothing-summer-has-not-gotten-latter-over-the-20-years.this-is-the-proccess-for-proving-it.}}

Setup:

\begin{Shaded}
\begin{Highlighting}[]
\KeywordTok{setwd}\NormalTok{(}\StringTok{"D:/ernie/self-study/GTxMicroMasters/Introduction to Analytics Modeling/week3/homework"}\NormalTok{)}
\KeywordTok{library}\NormalTok{(magrittr)}
\KeywordTok{library}\NormalTok{(tidyverse)}
\KeywordTok{library}\NormalTok{(lubridate)}
\KeywordTok{library}\NormalTok{(corrplot)}
\KeywordTok{library}\NormalTok{(leaps)}
\end{Highlighting}
\end{Shaded}

turning data into time. series format

\begin{Shaded}
\begin{Highlighting}[]
\NormalTok{weather <-}\StringTok{ }\KeywordTok{data.frame}\NormalTok{(}\KeywordTok{read.table}\NormalTok{(}\StringTok{"temps.txt"}\NormalTok{ , }\DataTypeTok{header =}\NormalTok{ T))}\OperatorTok
\StringTok{  }\KeywordTok{select}\NormalTok{(.,}\OperatorTok{-}\StringTok{ }\NormalTok{DAY) }\OperatorTok
\StringTok{  }\KeywordTok{unlist}\NormalTok{()}\OperatorTok
\StringTok{  }\KeywordTok{as.vector}\NormalTok{()}\OperatorTok
\StringTok{  }\KeywordTok{ts}\NormalTok{(}\DataTypeTok{start =} \DecValTok{1996}\NormalTok{ , }\DataTypeTok{end =} \DecValTok{2015}\NormalTok{ , }\DataTypeTok{frequency =} \DecValTok{100}\NormalTok{)}
\end{Highlighting}
\end{Shaded}

Graphically represented: It is hard to see actual trends

\begin{Shaded}
\begin{Highlighting}[]
\KeywordTok{plot}\NormalTok{(weather)}
\end{Highlighting}
\end{Shaded}

\includegraphics{week-3-homework_files/figure-latex/unnamed-chunk-3-1.pdf}

Putting down the HoltWinters Function

\begin{Shaded}
\begin{Highlighting}[]
\KeywordTok{HoltWinters}\NormalTok{(weather)}
\end{Highlighting}
\end{Shaded}

\begin{verbatim}
## Holt-Winters exponential smoothing with trend and additive seasonal component.
## 
## Call:
## HoltWinters(x = weather)
## 
## Smoothing parameters:
##  alpha: 0.7015953
##  beta : 0
##  gamma: 0.6813504
## 
## Coefficients:
##              [,1]
## a    102.12556505
## b     -0.01546985
## s1   -10.48519724
## s2   -12.44318212
## s3   -10.99764581
## s4   -12.75209947
## s5   -12.62974056
## s6   -11.29182605
## s7   -10.93658628
## s8    -5.40303859
## s9    -6.79845771
## s10   -3.39388173
## s11   -6.86132351
## s12   -6.30334825
## s13   -9.27608984
## s14   -8.60454327
## s15   -9.35894641
## s16  -11.98293559
## s17  -10.08152712
## s18   -5.73359214
## s19   -4.21236012
## s20   -6.33379226
## s21   -5.18509908
## s22   -1.14584131
## s23   -2.59594123
## s24   -2.24708371
## s25   -1.20692084
## s26    1.98058414
## s27    1.13684508
## s28    4.26424229
## s29    4.68465208
## s30    7.72445661
## s31    4.97002347
## s32    7.89317690
## s33    3.94494859
## s34    3.91773586
## s35    1.96066900
## s36    3.12215825
## s37    3.36315823
## s38    3.39629121
## s39    7.73828726
## s40    7.98122071
## s41    8.30522935
## s42    9.35335153
## s43    9.59278999
## s44    7.23771484
## s45    7.81844143
## s46    5.46541873
## s47    5.04169526
## s48    5.02802732
## s49    5.36288062
## s50    7.31875821
## s51    7.81973186
## s52    4.87303155
## s53    6.23219406
## s54    3.63600461
## s55    6.98631128
## s56    6.72090297
## s57    5.95574278
## s58    5.40068097
## s59    2.13451835
## s60    2.66065990
## s61    1.01986035
## s62    2.08035759
## s63    2.76693224
## s64    1.80908389
## s65    3.14082126
## s66    1.13910230
## s67    1.42982098
## s68    0.15996809
## s69    0.73405805
## s70    4.42836996
## s71    4.11636798
## s72    4.56871751
## s73    4.89836713
## s74    5.94492884
## s75    4.02735738
## s76    4.43237712
## s77    4.85538584
## s78    5.15469049
## s79   -0.32306323
## s80    2.61750562
## s81   -0.96168460
## s82    0.75065521
## s83    1.48361721
## s84   -1.28613918
## s85   -0.38311462
## s86    0.17466496
## s87    0.76124762
## s88    3.07396243
## s89    2.29772241
## s90   -0.13094343
## s91   -0.94184454
## s92   -1.16934790
## s93   -1.93458425
## s94   -5.57050542
## s95  -11.59120257
## s96   -7.82643736
## s97   -3.97388600
## s98   -7.74355298
## s99   -5.55287173
## s100  -7.58855363
\end{verbatim}

The main focus is here : Smoothing parameters: alpha: 0.7015953 beta : 0
gamma: 0.6813504

the beta of the Holt Winters Function is 0, indicating no overall trend,
which matches our intuition.

\hypertarget{question-8.1}{%
\subsection{Question 8.1}\label{question-8.1}}

\hypertarget{describe-a-situation-or-problem-from-your-job-everyday-life-current-events-etc.-for-which-a-linear-regression-model-would-be-appropriate.-list-some-up-to-5-predictors-that-you-might-use.}{%
\subsubsection{Describe a situation or problem from your job, everyday
life, current events, etc., for which a linear regression model would be
appropriate. List some (up to 5) predictors that you might
use.}\label{describe-a-situation-or-problem-from-your-job-everyday-life-current-events-etc.-for-which-a-linear-regression-model-would-be-appropriate.-list-some-up-to-5-predictors-that-you-might-use.}}

A good opportunity would be predicting a baseball team's winning
chances. It would be through parameters including: team average
ERA(Earned run average) team average batting average team average
slugging average and team average fielding percentage.

\hypertarget{question-8.2}{%
\subsection{Question 8.2}\label{question-8.2}}

\hypertarget{using-crime-data-use-regression-a-useful-r-function-islm-or-glm-to-predict-the-observed-crime-rate-in-a-cityin-a-city-with-the-following-data}{%
\subsubsection{Using crime data , use regression (a useful R function
islm or glm) to predict the observed crime rate in a cityin a city with
the following
data:}\label{using-crime-data-use-regression-a-useful-r-function-islm-or-glm-to-predict-the-observed-crime-rate-in-a-cityin-a-city-with-the-following-data}}

M = 14.0 So = 0 Ed = 10.0 Po1 = 12.0 Po2 = 15.5 LF = 0.640 M.F = 94.0
Pop = 150 NW = 1.1 U1 = 0.120 U2 = 3.6 Wealth = 3200 Ineq = 20.1 Prob =
0.04 Time = 39.0 Show your model (factors used and their coefficients),
the software output, and the quality of fit.

\hypertarget{answer-i-created-a-model-omitting-parameters-that-are-too-low-in-correlation-with-the-results-or-are-highly-correlated-with-other-parameters-making-them-un-independent.-the-model-i-created-has-an-75-r-squared-value-and-the-prediction-according-to-the-model-is-1177.978.}{%
\subsubsection{\texorpdfstring{Answer : I created a model omitting
parameters that are too low in correlation with the results or are
highly correlated with other parameters, making them un-independent. The
model I created has an 75\% R-squared value and the prediction according
to the model is 1177.978.
}{Answer : I created a model omitting parameters that are too low in correlation with the results or are highly correlated with other parameters, making them un-independent. The model I created has an 75\% R-squared value and the prediction according to the model is 1177.978.  }}\label{answer-i-created-a-model-omitting-parameters-that-are-too-low-in-correlation-with-the-results-or-are-highly-correlated-with-other-parameters-making-them-un-independent.-the-model-i-created-has-an-75-r-squared-value-and-the-prediction-according-to-the-model-is-1177.978.}}

Read Data

\begin{Shaded}
\begin{Highlighting}[]
\NormalTok{crime <-}\StringTok{ }\KeywordTok{read.table}\NormalTok{(}\StringTok{"uscrime.txt"}\NormalTok{ , }\DataTypeTok{header =} \OtherTok{TRUE}\NormalTok{)}\OperatorTok
\StringTok{  }\KeywordTok{data.frame}\NormalTok{()}
\end{Highlighting}
\end{Shaded}

Showing the correlation between predictors

\begin{Shaded}
\begin{Highlighting}[]
\NormalTok{pl1 <-}\StringTok{ }\KeywordTok{corrplot}\NormalTok{(}\KeywordTok{cor}\NormalTok{(crime))}
\end{Highlighting}
\end{Shaded}

\includegraphics{week-3-homework_files/figure-latex/unnamed-chunk-6-1.pdf}

We eliminate predictor P02 due to its high correlation with p01

Numeric and Graphical representation of correlation with the crime
variable

\begin{Shaded}
\begin{Highlighting}[]
\NormalTok{cor_relation <-}\StringTok{ }\KeywordTok{abs}\NormalTok{(}\KeywordTok{cor}\NormalTok{(crime}\OperatorTok{$}\NormalTok{Crime , crime[,}\DecValTok{1}\OperatorTok{:}\DecValTok{15}\NormalTok{]))}\OperatorTok
\StringTok{  }\KeywordTok{data.frame}\NormalTok{()}
\NormalTok{cor_relation <-}\StringTok{ }\NormalTok{cor_relation}\OperatorTok
\StringTok{  }\KeywordTok{gather}\NormalTok{(predictor, correlation)}
\NormalTok{cor_relation}
\end{Highlighting}
\end{Shaded}

\begin{verbatim}
##    predictor correlation
## 1          M  0.08947240
## 2         So  0.09063696
## 3         Ed  0.32283487
## 4        Po1  0.68760446
## 5        Po2  0.66671414
## 6         LF  0.18886635
## 7        M.F  0.21391426
## 8        Pop  0.33747406
## 9         NW  0.03259884
## 10        U1  0.05047792
## 11        U2  0.17732065
## 12    Wealth  0.44131995
## 13      Ineq  0.17902373
## 14      Prob  0.42742219
## 15      Time  0.14986606
\end{verbatim}

\begin{Shaded}
\begin{Highlighting}[]
\NormalTok{pl2 <-}\StringTok{ }\KeywordTok{ggplot}\NormalTok{(}\DataTypeTok{data =}\NormalTok{ cor_relation, }\KeywordTok{aes}\NormalTok{( }\DataTypeTok{x  =}\NormalTok{ predictor , }\DataTypeTok{y =}\NormalTok{ correlation , }\DataTypeTok{fill =}\NormalTok{ correlation )) }\OperatorTok{+}
\StringTok{  }\KeywordTok{geom_col}\NormalTok{()}
\NormalTok{pl2}
\end{Highlighting}
\end{Shaded}

\includegraphics{week-3-homework_files/figure-latex/unnamed-chunk-7-1.pdf}

We remove NW,U1 and So due to low correlation

Constructing the model:

\begin{Shaded}
\begin{Highlighting}[]
\NormalTok{model <-}\StringTok{ }\KeywordTok{lm}\NormalTok{ (}\DataTypeTok{data =}\NormalTok{ crime , Crime }\OperatorTok{~}\StringTok{ }\NormalTok{Ed }\OperatorTok{+}\StringTok{ }\NormalTok{Ineq }\OperatorTok{+}\StringTok{ }\NormalTok{LF }\OperatorTok{+}\StringTok{ }\NormalTok{M }\OperatorTok{+}\StringTok{ }\NormalTok{M.F }\OperatorTok{+}\NormalTok{Po1 }\OperatorTok{+}\StringTok{ }\NormalTok{Pop }\OperatorTok{+}\StringTok{ }\NormalTok{Prob  }\OperatorTok{+}\StringTok{ }\NormalTok{Time }\OperatorTok{+}\StringTok{ }\NormalTok{Ed)}
\KeywordTok{summary}\NormalTok{(model)}
\end{Highlighting}
\end{Shaded}

\begin{verbatim}
## 
## Call:
## lm(formula = Crime ~ Ed + Ineq + LF + M + M.F + Po1 + Pop + Prob + 
##     Time + Ed, data = crime)
## 
## Residuals:
##     Min      1Q  Median      3Q     Max 
## -468.62 -100.73   -6.44  139.91  520.35 
## 
## Coefficients:
##               Estimate Std. Error t value Pr(>|t|)    
## (Intercept) -5189.5782  1460.9341  -3.552 0.001063 ** 
## Ed            140.9730    57.8900   2.435 0.019823 *  
## Ineq           68.7477    15.8765   4.330 0.000109 ***
## LF           -609.2340  1065.0117  -0.572 0.570751    
## M              68.3357    35.3331   1.934 0.060784 .  
## M.F            17.8666    15.2790   1.169 0.249738    
## Po1           126.5215    17.3893   7.276 1.22e-08 ***
## Pop            -0.6526     1.2716  -0.513 0.610833    
## Prob        -4006.6838  2033.8562  -1.970 0.056359 .  
## Time            1.7858     6.6248   0.270 0.788995    
## ---
## Signif. codes:  0 '***' 0.001 '**' 0.01 '*' 0.05 '.' 0.1 ' ' 1
## 
## Residual standard error: 213.8 on 37 degrees of freedom
## Multiple R-squared:  0.7543, Adjusted R-squared:  0.6945 
## F-statistic: 12.62 on 9 and 37 DF,  p-value: 7.275e-09
\end{verbatim}

The summary results show that R- squared is at about 75\% which is a
good enough result. One thing to notice is however the Variables
LF,M.F., Pop and Time are not statistically significant. Nevertheless,
We test out the model using the given numbers:

\begin{Shaded}
\begin{Highlighting}[]
\NormalTok{test <-}\StringTok{ }\KeywordTok{data.frame}\NormalTok{(   }\DataTypeTok{M =} \FloatTok{14.0}\NormalTok{,}\DataTypeTok{So =} \DecValTok{0}\NormalTok{,}\DataTypeTok{Ed =} \FloatTok{10.0}\NormalTok{,}\DataTypeTok{Po1 =} \FloatTok{12.0}\NormalTok{,}
                      \DataTypeTok{Po2 =} \FloatTok{15.5}\NormalTok{,}\DataTypeTok{LF =} \FloatTok{0.640}\NormalTok{,}\DataTypeTok{M.F =} \FloatTok{94.0}\NormalTok{ ,}\DataTypeTok{Pop =} \DecValTok{150}\NormalTok{,}
                      \DataTypeTok{NW =} \FloatTok{1.1}\NormalTok{,}\DataTypeTok{U1 =} \FloatTok{0.120}\NormalTok{,}\DataTypeTok{U2 =} \FloatTok{3.6}\NormalTok{,}\DataTypeTok{Wealth =} \DecValTok{3200}\NormalTok{,}\DataTypeTok{Ineq =} \FloatTok{20.1}\NormalTok{,}\DataTypeTok{Prob =} \FloatTok{0.04}\NormalTok{,}\DataTypeTok{Time =} \FloatTok{39.0}
\NormalTok{                      )}
\KeywordTok{predict}\NormalTok{(model,test)}
\end{Highlighting}
\end{Shaded}

\begin{verbatim}
##        1 
## 1177.978
\end{verbatim}

Therefore my prediction is 1177.978.

\end{document}
